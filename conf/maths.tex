% we assume that amsmath/mathtools is loaded

% formulism stuff
% remember to use stared `*' version as these will use \left \right auto-sizing for delimiters.
%% should not be called directly
\DeclarePairedDelimiterX{\transform}[1]{\lbrace}{\rbrace}{\mathopen{}#1}
\DeclarePairedDelimiterX{\tresult}[1]{\lbrace}{.}{\;#1\;}
\DeclarePairedDelimiterX{\tuple}[1]{\langle}{\rangle}{\,#1\,}
\DeclarePairedDelimiterX{\collection}[1]{\llangle}{\rrangle}{\;#1\;}
\DeclarePairedDelimiterX{\group}[1]{\lgroup}{\rgroup}{\;#1\;}
\DeclarePairedDelimiterX{\fun}[1]{\lparen}{\rparen}{\;#1\;}
\newcommand{\listitem}[1]{&\bm{:}#1\\}
\NewDocumentCommand{\alignedlist}{>{\SplitList{,}}m}{\begin{aligned}\ProcessList{#1}{\listitem}\end{aligned}}
\newcommand{\envitem}[1]{\ #1}
\NewDocumentCommand{\env}{>{\SplitList{,}}m}{\ProcessList{#1}{\envitem}}
%% primary formalism
%% XXX we need to phase these two out
\NewDocumentCommand{\PHASE}{omm}{\mathrm{#2}\;\IfNoValueTF{#1}{\transform*{#3}}{\transform*{#3}\alignedlist{#1}}}
\NewDocumentCommand{\TRANS}{m}{\tresult*{#1}}
%% new formalism
\NewDocumentCommand{\nPHASE}{omm}{\mathbf{#2}\IfNoValueTF{#1}{\transform*{#3}}{\transform*{#3}_{\env{#1}}}}
\NewDocumentCommand{\nTRANS}{m}{#1}
\NewDocumentCommand{\nVAL}{mm}{#1\!:\!\mathtt{#2}}
\NewDocumentCommand{\nCAT}{mm}{#1\!\mathrel{+}\mathrel{+}\!#2}
\NewDocumentCommand{\nPLUS}{mm}{#1\!\mathrel{\oplus}\!#2}
\NewDocumentCommand{\APPLY}{omm}{\mathcal{#2}\;\IfNoValueTF{#1}{\function*{#3}}{\functionap*{#3}{#1}}}
\NewDocumentCommand{\TUPLE}{m}{\hookrightarrow\collection*{\alignedlist{#1}}} % code-attributes
\NewDocumentCommand{\COLL}{m}{\tuple*{#1}} % code-attributes
\newcommand{\sem}[1]{\llbracket#1\rrbracket}
\newcommand{\shp}[1]{\ensuremath{\big\lvert#1\big\rvert}}
